%
% Masters/Doctoral Thesis 
% LaTeX Template
% Version 2.3 (25/3/16)
%
% This template has been downloaded from:
% http://www.LaTeXTemplates.com
%
% Version 2.x major modifications by:
% Vel (vel@latextemplates.com)
%
% This template is based on a template by:
% Steve Gunn (http://users.ecs.soton.ac.uk/srg/softwaretools/document/templates/)
% Sunil Patel (http://www.sunilpatel.co.uk/thesis-template/)
%
% Template license:
% CC BY-NC-SA 3.0 (http://creativecommons.org/licenses/by-nc-sa/3.0/)
%
%%%%%%%%%%%%%%%%%%%%%%%%%%%%%%%%%%%%%%%%%

%----------------------------------------------------------------------------------------
%	PACKAGES AND OTHER DOCUMENT CONFIGURATIONS
%----------------------------------------------------------------------------------------

\documentclass[
11pt, % The default document font size, options: 10pt, 11pt, 12pt
%oneside, % Two side (alternating margins) for binding by default, uncomment to switch to one side
%chapterinoneline,% Have the chapter title next to the number in one single line
%english, % ngerman for German
spanish,
singlespacing, % Single line spacing, alternatives: onehalfspacing or doublespacing
%draft, % Uncomment to enable draft mode (no pictures, no links, overfull hboxes indicated)
%nolistspacing, % If the document is onehalfspacing or doublespacing, uncomment this to set spacing in lists to single
%liststotoc, % Uncomment to add the list of figures/tables/etc to the table of contents
%toctotoc, % Uncomment to add the main table of contents to the table of contents
parskip, % Uncomment to add space between paragraphs
%nohyperref, % Uncomment to not load the hyperref package
headsepline, % Uncomment to get a line under the header
]{MastersDoctoralThesis} % The class file specifying the document structure

\usepackage{amsmath, amsthm, amssymb} % Pongo para las ecuaciones
\usepackage[utf8]{inputenc} % Required for inputting international characters
\usepackage[T1]{fontenc} % Output font encoding for international characters


\usepackage{palatino} % Use the Palatino font by default
%,style=authoryear
\usepackage[backend=bibtex,natbib=true]{biblatex} % Use the bibtex backend with the authoryear citation style (which resembles APA)

\addbibresource{references.bib} % The filename of the bibliography

\usepackage[autostyle=true]{csquotes} % Required to generate language-dependent quotes in the bibliography

\usepackage{caption}
\usepackage{subcaption}

%------------------------
\usepackage{listings}

%\usepackage[hyphens]{url}
%\usepackage[hidelinks]{hyperref}
%\hypersetup{breaklinks=true}
\urlstyle{same}
%\usepackage{cite}

%--------------------------

\usepackage{color}

%
%----------------------------------------------------------------------------------------
%	MARGIN SETTINGS
%----------------------------------------------------------------------------------------

\geometry{
	paper=a4paper, % Change to letterpaper for US letter
	inner=2cm, % Inner margin
	outer=3.3cm, % Outer margin
	bindingoffset=2cm, % Binding offset
	top=1.5cm, % Top margin
	bottom=1.5cm, % Bottom margin
	%showframe,% show how the type block is set on the page
}

%----------------------------------------------------------------------------------------
%	INFORMACIÓN DE LA MEMORIA
%----------------------------------------------------------------------------------------

\thesistitle{Diseño e implementación de un filtro CFAR con automatismos para un procesador de señales radar con tecnología SoC-FPGA} % El títulos de la memoria, se usa en la carátula y se puede usar el cualquier lugar del documento con el comando \ttitle
\supervisor{Ing. Oscar Guillermo Lombardero} % El nombre del director, se usa en la carátula y se puede usar el cualquier lugar del documento con el comando \supname
\degree{Ingeniería en electrónica} % Nombre del grado, se usa en la carátula y se puede usar el cualquier lugar del documento con el comando \degreename
\author{González, Fernando Augusto} % Tu nombre, se usa en la carátula y se puede usar el cualquier lugar del documento con el comando \authorname
\juradoUNO{Nombre del jurado 1 (pertenencia)} % Nombre y pertenencia del un jurado se usa en la carátula y se puede usar el cualquier lugar del documento con el comando \jur1name
\juradoDOS{Nombre del jurado 2 (pertenencia)} % Nombre y pertenencia del un jurado se usa en la carátula y se puede usar el cualquier lugar del documento con el comando \jur2name
\juradoTRES{Nombre del jurado 3 (pertenencia)} % Nombre y pertenencia del un jurado se usa en la carátula y se puede usar el cualquier lugar del documento con el comando \jur3name
\fechaINICIO{noviembre de 2019}
\fechaFINAL{agosto de 2020}

\subject{Informe de Proyecto Final de la Carrera de Ingeniería en Electrónica de la UNNE} % Your subject area, this is not currently used anywhere in the template, print it elsewhere with \subjectname
\keywords{CESE, Sistemas Embebidos, CIAA} % Keywords for your thesis, this is not currently used anywhere in the template, print it elsewhere with \keywordnames
\university{Universidad Nacional del Nordeste} % Your university's name and URL, this is used in the title page and abstract, print it elsewhere with \univname
\faculty{{Facultad de Ciencias Exactas y Naturales Y Agrimensura}} % Your faculty's name and URL, this is used in the title page and abstract, print it elsewhere with \facname
\department{Departamento de Ingeniería} % Your department's name and URL, this is used in the title page and abstract, print it elsewhere with \deptname
\group{{Laboratorio de Sistemas Embebidos}} % Your research group's name and URL, this is used in the title page, print it elsewhere with \groupname


\hypersetup{pdftitle=\ttitle} % Set the PDF's title to your title
\hypersetup{pdfauthor=\authorname} % Set the PDF's author to your name
\hypersetup{pdfkeywords=\keywordnames} % Set the PDF's keywords to your keywords


\newcaptionname{spanish}{\acknowledgementname}{Agradecimientos}
\newcaptionname{spanish}{\authorshipname}{Declaración de Autoría}
\newcaptionname{spanish}{\abbrevname}{Glosario}
\newcaptionname{spanish}{\byname}{por}

\renewcommand{\lstlistingname}{Algoritmo}% Listing -> Algorithm
\renewcommand{\lstlistlistingname}{Índice de \lstlistingname s}% List of Listings -> List of Algorithms

\renewcommand{\listtablename}{Índice de Tablas}
\renewcommand{\tablename}{Tabla} 

\addtolength{\footnotesep}{2mm} % Espacio adicional en los footnotes

\begin{document}

\frontmatter % Use roman page numbering style (i, ii, iii, iv...) for the pre-content pages

\pagestyle{plain} % Default to the plain heading style until the thesis style is called for the body content

%----------------------------------------------------------------------------------------
%	CARÁTULA
%----------------------------------------------------------------------------------------

\begin{titlepage}
\begin{center}

{\scshape\LARGE UNIVERSIDAD NACIONAL DEL NORDESTE\par}\vspace{0.1cm} % University name
{\scshape\LARGE FACULTAD DE CIENCIAS EXACTAS Y NATURALES Y AGRIMENSURA\par}\vspace{0.1cm} % Faculty name
{\scshape\LARGE Carrera de Ingeniería en Electrónica\par}\vspace{1cm} % Thesis type

\includegraphics[width=.3\textwidth]{./Figures/logo_unne.png}
\vspace{1cm}

\textsc{\Large Informe del Proyecto Final de Carrera}\\[0.5cm] % Thesis type

{\huge \bfseries \ttitle\par}\vspace{0.4cm} % Thesis title

\vspace{1cm}
\LARGE\textbf{Autor:\\
\authorname}\\ % Author name

\vspace{1cm}

\large
\vspace{10pt}
{Director:} \\
{\supname} % Supervisor name
 
%\vspace{1cm}
%Jurados:\\
%\jurunoname\\
%\jurdosname\\
%\jurtresname
 
\vfill
\textit{Este trabajo fue realizado en la Ciudad Corrientes y Resistencia, entre \fechaINICIOname \hspace{1pt} y \fechaFINALname.}
\end{center}
\end{titlepage}


%----------------------------------------------------------------------------------------
%	RESUMEN - ABSTRACT 
%----------------------------------------------------------------------------------------

\begin{abstract}
\addchaptertocentry{\abstractname} % Add the abstract to the table of contents
%
%The Thesis Abstract is written here (and usually kept to just this page). The page is kept centered vertically so can expand into the blank space above the title too\ldots
\centering

El presente trajo consiste en la arquitectura, diseño e implementación de un sistema CFAR que utiliza los algoritmos CA-CFAR, SOCA-CFAR y GOCA-CFAR, y de un sistema configurador que divide la cobertura en 32 sectores y almacena en cada sector una configuración diferente a utilizar por el CFAR y otras etapas de procesamiento posterior.

El diseño de ambos sistemas fue realizado utilizando conceptos de circuitos lógicos, procesamiento de señales y diseño digital entre otros. Para la realización se empleó el lenguaje de descripción de hardware VHDL.

La implementación de los circuitos se llevó a cabo en una placa ADC-SoC, la cual contiene, entre otras cosas, conversores analógico a digital de alta velocidad y un SoC-FPGA. En este último, se cuenta con un sistema de procesamiento duro (\textit{Hard Processor System}, HPS) ARM Cortex-A9.

Los datos procesados por el CFAR fueron de un radar FPS113. Los datos recibidos por el configurador provienen de sistema operativo Linux embebido que se comunica con la FPGA a través de un puente HPS.

\end{abstract}

%----------------------------------------------------------------------------------------
%	CONTENIDO DE LA MEMORIA  - AGRADECIMIENTOS
%----------------------------------------------------------------------------------------

\begin{acknowledgements}
%\addchaptertocentry{\acknowledgementname} % Descomentando esta línea se puede agregar los agradecimientos al índice
\vspace{1.5cm}

Agradecimientos personales. \textbf{[OPCIONAL]} 
%A todos los profesores que tuve durante la carrera, quienes me han formado.
%A mis compañeros y amigos, en especial a Leandro Torrent, Martín Duarte y Diego Aceval, por todos los momentos compartidos.
%A Nadia por acompañarme en el último tramo de la carrera, donde se realizan los últimos esfuerzos y los más difíciles.
%A Miguel Eduardo del Valle Camino, por su gran compañerismo y por enseñarme. Su predisposición, compañerismo, apoyo y confianza fueron importantes para el desarrollo de este proyecto.
%Finalmente a mi familia, por animarme a tomar estudiar esta carrera y apoyarme en todo momento.

\end{acknowledgements}

%----------------------------------------------------------------------------------------
%	LISTA DE CONTENIDOS/FIGURAS/TABLAS
%----------------------------------------------------------------------------------------
\renewcommand{\listtablename}{Índice de Tablas}

\tableofcontents % Prints the main table of contents

\listoffigures % Prints the list of figures

%\listoftables % Prints the list of tables


%----------------------------------------------------------------------------------------
%	CONTENIDO DE LA MEMORIA  - DEDICATORIA
%----------------------------------------------------------------------------------------

%\dedicatory{\textbf{Dedicado a mi madre y mi padre}}  % escribir acá si se desea una dedicatoria

%----------------------------------------------------------------------------------------
%	CONTENIDO DE LA MEMORIA  - CAPÍTULOS
%----------------------------------------------------------------------------------------

\mainmatter % Begin numeric (1,2,3...) page numbering

\pagestyle{thesis} % Return the page headers back to the "thesis" style

\renewcommand{\tablename}{Tabla} 

% Incluir los capítulos como archivos separados desde la carpeta Chapters
% Descomentar las líneas a medida que se escriben los capítulos

\include{Chapters/Chapter1}
\chapter{Materiales y metodos} % Main chapter title

\label{Chapter2}

%----------------------------------------------------------------------------------------
%	SECTION 1
% Aquí se mencionan los componentes que se utilizaron y la metodología empleada.
% Luego pueden ir las plaquetas que se diseñaron para el prototipo, esquemáticos,
% diagrama de flujo, dibujos en 3D, etc.
%----------------------------------------------------------------------------------------

\section{Dispositivos y herramientas utilizadas}

Para la implementación del presente trabajo se utilizó:
\begin{itemize}
\item Placa ADC-SoC de TerasIC.
\item Software Quartus Prime.
\item Software ModelSim.
\item Extractor analógico.
\item Simulador de radar.

\end{itemize}

\subsection{Placa ADC-SoC de TerasIC}
Es una placa tipo SoC-FPGA. El circuito de conversión analógica a digital incorporado utiliza conectores SMA como interfaz de entrada y proporciona dos canales de conversión, cada uno de 14-bits de resolución y una frecuencia de muestreo de hasta 150 MSPS (Megasamples per Second).

\begin{figure}
\centering
\includegraphics[scale=0.15]{./Figures/ADC-SoC.jpg}
\caption{Placa ADC-SoC de TerasIC}
\end{figure}

El siguiente hardware se proporciona en la placa:

\begin{itemize}
\item 	FPGA
	\begin{itemize}
	\item Dispositivo Altera Cyclone® V.
	\item Dispositivo de configuración en serie.
	\item USB-Blaster II integrado para programación; Modo JTAG
	\item 2 pulsadores
	\item 4 interruptores deslizantes
	\item 8 LED de usuario verdes
	\item Tres fuentes de reloj de 50 MHz del generador de reloj
	\item Un cabezal de expansión de 40 pines
	\item Un encabezado de expansión Arduino (compatibilidad con Arduino Uno R3) donde se 					puede conectar con los 'shields' Arduino.
	\item Un encabezado de expansión de entrada analógica de 10 pines (compartido con la 				entrada analógica Arduino).
	\item Convertidor A / D, interfaz SPI de 4 pines con FPGA
	\item  Dos convertidores AD de 14 bits con 150 MSPS (megamuestras por segundo)
	\end{itemize}
\item HPS (Hard Processor System)
	\begin{itemize}
	\item Procesador ARM Cortex-A9 de doble núcleo de 925 MHz
	\item 1GB DDR3 SDRAM (bus de datos de 32 bits)
	\item 1 Gigabit Ethernet PHY con conector RJ45
	\item Puerto USB OTG, conector USB Micro-AB
	\item Toma de tarjeta micro SD
	\item Acelerómetro (interfaz I2C + interrupción)
	\item UART a USB, conector USB Mini-B
	\item Botón de reinicio en caliente y botón de reinicio en frío
	\item Un botón de usuario y un LED de usuario
	\item Cabecera de expansión LTC 2x7
	\item RTC integrado (reloj en tiempo real)
	\end{itemize}
\end{itemize}


\begin{figure}
\centering
\includegraphics[scale=0.25]{./Figures/ADC-SoC_blockdiagram.jpg}
\caption{Diagrama en bloques de la placa ADC-SoC de TerasIC}
\end{figure}


En el chip Cyclone V de Altera (propiedad de Intel) se integra una FPGA y un HPS unidos por un puente HPS. Por defecto, la tarjeta micro SD tiene instalado el SO “​Linux Yocto Poky 8.0​”, el cual permite correr programas compilados en lenguaje C/C++ entre otros.
 
 
\subsection{Quartus Prime}
Quartus Prime es una herramienta de software producida por Altera para el análisis y la síntesis de diseños realizados en HDL. Permite compilar diseños, realizar análisis temporales, examinar diagramas RTL y configurar el dispositivo de destino con el programador.

Proporciona herramientas para trabajar en diferentes fases del diseño en FPGA como la creación del diseño, el agregado de restricciones, la compilación, el análisis de tiempos y la configuración de la FPGA con un promagrador. Se describen las características de Quartus a continuación:

\begin{itemize}
\item
Creación del diseño: Es posible diseñar en nivel RTL con lenguajes VHDL, Verilog o SystemVerilog. Además posee una herramienta llamada Platform Designer que crea automáticamente la lógica de interconexión a partir de la conectividad de alto nivel que se especifique. La automatización de interconexión elimina la laboriosa tarea de especificar conexiones HDL a nivel del sistema. De esta manera se pueden  especificar los requisitos de la interfaz e integrar componentes de IP dentro de una representación gráfica del sistema. En el presente proyectó se utilizó Platform Designer para configurar el puente HPS.
\item
Permite el agregado de restricciones al diseño con herramientas denominadas assigment editor y pin planner.
\item
Permite compilar el diseño, compuesto por las siguientes etapas:
\begin{itemize}
	\item
	Análisis y Síntesis
	Se evalúa el código para detectar la correcta escritura del mismo y se verifica que el mismo sea sintetizable, es decir que pueda ser implementado con lógica.
	\item
	Fitter (Colocación y Ruteo, Place \& Route): Se asignan recursos específicos de la FPGA para cumplir con el diseño. En etapa etapa además la herramienta utiliza la información sobre las restricciones de tiempo evaluar qué celdas utilizar (en terminos de distancia). Por otro lado, luego de la colocación, se emplean técnicas de optimización en la asignación de recursos.
	\item
	Generación de archivos de programación: Se generan los archivos necesarios para programar la FPGA.
	\item
	Análisis de tiempos: Se verifica que se cumplan con las restricciones de tiempo especificadas en el diseño. Por ejemplo los referidos a la generación de señales de reloj internas, derivadas de fuentes de señales de reloj externas.
	\end{itemize}
	
\item 
Posee una herramienta llamada Signal Tab Logic Analyzer que permite realizar un debug mientras el sistema se ejecuta en la FPGA.

\end{itemize}

\subsubsection{Visores de netlist}
A medida que los diseños de FPGA crecen en tamaño y complejidad, la capacidad de analizar, depurar, optimizar y restringir el diseño es fundamental. Quartus posee visores llamados RTL Viewer, State Machine Viewer y Technology Map Viewer. Cada uno permite ver representaciones esquemáticas de la estructura interna del diseño. Cada visor muestra una vista única del netlist (lista de redes) que se producen en diferentes etapas de la compilación, como se ilustra en la figura \ref{fig:netlist_viewers}. Se describen a continuación:

\begin{figure}
\centering
\includegraphics[scale=0.5]{./Figures/netlist_viewers.png}
\caption{Ubicación de los visores de netlist en el flujo de diseño de Quartus}
\label{fig:netlist_viewers}
\end{figure}

\begin{itemize}
\item
RTL Viewer:
Permite ver un esquema de la lista de conexiones de diseño después de la etapa de Análisis y elaboración y de la extracción de la lista de conexiones, pero antes de la síntesis y las optimizaciones de ajuste. Esta vista no es la estructura final del diseño, porque no se incluyen todas las optimizaciones; en cambio, es la vista más cercana posible al diseño original. Si el diseño utiliza síntesis integrada, esta vista muestra cómo el Quartus interpreta los archivos de diseño; Si está utilizando una herramienta de síntesis de EDA de terceros, esta vista muestra la lista de conexiones escrita por la herramienta de síntesis de EDA.

\item
State Machine Viewer
Proporciona una vista de alto nivel de las máquinas de estados finitos en el diseño y muestra la estructura interna de las máquinas de estados en el diseño, incluida una vista más detallada de la entrada y salida de los nodos de estado individuales. También muestra las transiciones de los nodos en formato de tabla.

\item
Technology Map Viewer
Permite ver un esquema de bajo nivel específico de la tecnología de la lista de redes de diseño después del ajuste o después de Análisis y síntesis. Puede acceder a la vista del esquema posterior al ajuste (post-fitting) o posterior al mapeo (post-mapping), independientemente de la herramienta de síntesis que se utilice. Cuando se abre desde una ruta de tiempo en el informe del Analizador de tiempo, el Visor de mapas de tecnología también muestra información detallada sobre el retardo de tiempo para la ruta de tiempo.


\end{itemize}


\subsection{ModelSim}

La simulación es un paso crítico en el diseño para FPGA y ASIC. Permite al diseñador estimular su diseño y ver cómo el código que escribió reacciona al estímulo. Una gran simulación contendrá todos los estados posibles del diseño para garantizar que todos los escenarios de entrada se manejen de manera adecuada. Este ejercicio permite descubrir si en alguna parte del diseño, por ejemplo en procesos combinacionales,  no se contempló todos casos posibles y en caso de descubrir un comportamiento no esperado se procede a corregirlo.

Para este proyecto se utilizó ModelSim, un entorno de simulación creado por Mentor Graphics. ModelSim Está diseñado para trabajar con Verilog y VHDL. Además cuenta con un debugger, el cual se puede emplear, por ejemplo, para descubrir comportamientos más difíciles de dilucidar que no son determinados a simple vista. 
\chapter{Ensayos} % Main chapter title

\label{Chapter3} % Change X to a consecutive number; for referencing this chapter elsewhere, use \ref{ChapterX}


%----------------------------------------------------------------------------------------
%	Aquí se describen los distintos ensayos que se realizaron, y bajo que protocolo se
% trabajó. Mencionar los instrumentos utilizados.
% Se pueden incluir los ensayos realizados con los prograparticular aquellos que realizan curvas paramétricas.

%----------------------------------------------------------------------------------------
\section{Ensayos con marcos de prueba}
\label{Sección: Ensayos con marcos de prueba}
Se realizó el \textit{testbench} (marco de prueba) para cada entidad de diseño dentro del proyecto, para ensayar la respuesta de cada módulo a señales estímulo. Estas señales fueron representadas de manera de emular las señales correspondientes de aquellas para las que cada módulo se diseñó para manejar. Este tipo de ensayos se utilizó para observar la respuesta a estímulos de las diferentes entidades de la jerarquía de diseño yendo desde la entidad en el nivel más bajo a la entidad en el nivel más alto en la jerarquía.


\begin{itemize}
\item Jerarquía del CFAR:
	\begin{enumerate}
	\item FFD/Celda test
	\item
		\begin{enumerate}
		\item Celdas de guarda
		\item Celdas de referencia
		\end{enumerate}
	\item Comparador
	\item CFAR
	\item Contador CFAR
	\end{enumerate}
 
\item Jerarquía del Configurador
	\begin{enumerate}
	\item
		\begin{enumerate}
		\item Registro de entrada
		\item Registro de salida
		\end{enumerate}
	\item Sector fijo
	\item Procesador de presencias
	\item Ajuste de multiplicador
	\item Sector
	\item Sectorizador
	\item Configurador
	\end{enumerate}

\end{itemize}


\subsubsection{Flujo de simulación básico}


\begin{figure}
\centering
\includegraphics[scale=.7]{./Figures/flujo_simulacion.png}
\caption{Flujo de simulación básico}
\label{flujo_simulacion}
\end{figure}

El diagrama de la figura \ref{flujo_simulacion} muestra los pasos básicos para simular un diseño en ModelSim.

\begin{itemize}
\item
Creación de la biblioteca de trabajo:
En \textit{ModelSim}, todos los diseños se compilan en una biblioteca. Normalmente, una nueva simulación en \textit{ModelSim} comienza creando una biblioteca de trabajo llamada "trabajo", que es el nombre de biblioteca por defecto que utiliza el compilador como destino predeterminado para las unidades de diseño compiladas. Se creó, de este modo, dos bibliotecas, una para el cfar y otra para el configurador, cada una con todas las entidades de la jerarquía y sus correspondiente \textit{testbench}.

\item
Compilación del diseño:
Después de crear la biblioteca de trabajo, se compiló las unidades de diseño en ella. Esta compilación analiza cada archivo de forma individual y reporta, por ejemplo, errores de sintaxis o de falta de entidades en la librería que fueron instanciados en la entidad compilada. Estos reportes permiten realizar las correcciones necesarias en la entidad y/o arquitectura de un determinado archivo fuente. Sin embargo la compilación de todas las entidades no asegura el funcionamiento de todas ellas trabajando en conjunto (como un sistema) debido a que el análisis es a nivel individual.

\item
Carga del simulador con el diseño y ejecución de la simulación con el diseño compilado:
Una vez que la compilación resultó exitosa para todas las entidades de cada jerarquía se cargó el simulador con el \textit{testbench} asociado a cada entidad según el orden propuesto de ensayos descrito en la sección \ref{Sección: Ensayos con marcos de prueba}. Luego se estableció el tiempo de simulación en cero y se ejecutó en cada caso la simulación.

\item
Depuración sus resultados:
Se evaluó los resultados para cada simulación.
En algunos casos se escribió reportes dentro de la arquitectura VHDL por ejemplo, para determinar si en un \textit{process} se ingresó correctamente en todos los casos contemplados de alguna sentencia condicional. En otros se visualizó las formas de onda de las señales estímulo (entrada) y de respuesta (salida). En casos más complejos en donde no se obtuvo los resultados esperados y no se pudo determinar de "primera mano" la explicación a un determinado comportamiento, se utilizó el \textit{debugger} incorporado en \textit{ModelSim} para recorrer la ejecución en cada sentencia de las arquitecturas involucradas en el archivo fuente en cuestión.
\end{itemize}



    
    
   
    



\section{Ensayos con visores de netlist}

Se utilizó los visores de netlist \textit{RTL Viewer} y \textit{Technology Map Viewer}, descritos en la seccion \ref{seccion: Visores de netlist} del capítulo 2, para verificar la arquitectura de diseño de las entidades mencionadas en la sección \ref{Sección: Ensayos con marcos de prueba}.

El visor \textit{RTL Viewer} se utilizó para ver los resultados de la síntesis inicial y determinar si se creó la lógica necesaria y si el software interpretó correctamente la lógica y las conexiones. De esta manera se verificó el diseño visualmente.

En los casos en los que la simulación RTL arrojó un comportamiento inesperado, este visor fue útil para rastrear la lista de conexiones y asegurarse que las conexiones y la lógica en el diseño sean las esperadas. En caso de verificar la correcta interpretación de la herramienta de síntesis a este nivel, se analizó las etapas posteriores del proceso de diseño para investigar posibles violaciones de tiempo o problemas en el flujo de verificación en sí.

El visor \textit{Technology Map Viewer} se utilizó para ver los resultados al final de la compilación completa del diseño y determinar qué recursos fueron asignados en aquellos bloques en los que se observó un comportamiento erróneo durante la simulación en \textit{ModelSim}. Además, se utilizó para rastear conexiones entre entidades de diseño, sobre todo para verificar aquellas conexiones dedicadas a transportar la señal de un reloj a varios componentes.




\section{Ensayos con simulador y decodificador}
Para verificar el funcionamiento del Configurador se utilizó un simulador. Dicho equipo se integró dentro de la FPGA con el objetivo de poder tener dos modos de configuración: uno en donde la señal a analizar proveniera del radar y otro en el que la señal proveniera del simulador. En el segundo modo de configuración, el equipo simuló todas las señales correspondientes a los radares primario y secundario, incluidos los ecos de estos sensores. Por defecto, en este modo se simulan 8 plots primarios y 8 plots secundarios por cada giro de antena, como se ilustra en la figura \ref{fig: plots simulador}. Este modo permite efectuar ensayos de mantenimiento y calibración sobre las distintas partes del Procesador Monoradar cuando no se dispone físicamente del sistema de radar.

Para ensayar el funcionamiento del configurador se diseñó un decodificador para recibir los datos de salida del mismo. Se dividió la cobertura en dos partes (agrupando 16 sectores en cada parte) y cada una se configuró con valores diferentes de algorítmo, multiplicador y coeficiente de ventana deslizante para verificar visualmente la confirmación (o no) de estos valores mediante el encendido (o apagado) de unos leds integrados en la placa.

Para determinar el crecimiento de la cantidad de targets detectados por el CFAR en un giro de antena, se utilizó el contador descrito en la sección \ref{subsubseccion: Contador CFAR}. Los cuatro bits más significativos de la cuenta se conectaron a cuatro leds integrados en la placa para verificar visualmente su variación a medida que el radar barrió la cobertura.

\begin{figure}
\centering
\includegraphics[scale=0.5]{./Figures/plots_simulador.png}
\caption{Distribución de plots de prueba}
\label{fig: plots simulador}
\end{figure}


\section{Ensayos con Procesador Monoradar preexistente}
Se utilizó un Procesador Monoradar preexistente para contrastar los datos arrojados por el Procesador Monoradar del que tanto el CFAR como el configurador descritos en este trabajo fueron parte. Para el CFAR se contrastó las ubicaciones de los plots detectados por ambos sistemas y se observó además la distribución de clutter. En el caso del configurador, se configuró varios sectores con diferentes valores de multiplicador para observar variación de clutter debido a los diferentes umbrales generados en estos sectores de la cobertura. Además para el configurador se varió el nivel de multiplicador  para algunos sectores, desde el nivel mínimo al máximo con el objetivo de visualizar en pantalla los límites de estos sectores.







% Chapter Template

\chapter{Resultados} % Main chapter title

\label{Chapter4} % Change X to a consecutive number; for referencing this chapter elsewhere, use \ref{ChapterX}

%----------------------------------------------------------------------------------------
%	SECTION 1
%----------------------------------------------------------------------------------------

\section{Resultados}
\label{sec: Resultados}

\subsection{Resultados del CFAR a ensayos con marcos de prueba}

\subsubsection{CFAR. Ensayo 1}
\label{Subsec:CFAR. Ensayo 1}

Se simuló el \textit{test bench} del CFAR para un intervalo de tiempo comprendido entre $0 us$ y $121 us$. La señal de estímulo del video de entrada se creó con las siguientes características para emular de manera aproximada una señal real:

\begin{itemize}
\item
Valor bajo equivalente al piso de ruido durante todo el periodo, excepto durante un pequeño intervalo de tiempo en la que adquiere un valor mayor. Se eligió como valor bajo \texttt{10000000000000} debido a que el conversor de alta velocidad es de 14-bit signado y el valor mencionado corresponde al cero. Físicamente el piso de ruido era cercano a cero. El valor de amplitud del pulso de video se eligió \texttt{10111011100000} como valor cercano al $50\%$ del rango de operación.


\item
Periódica de periodo suficientemente largo para que exista un solo pulso en el intervalo de tiempo que comprende la totalidad de celdas CFAR. La totalidad de las celdas CFAR generan un retardo indicado en la ecuación \ref{Eq:delay_cfar}, donde $N_{rc}$ es igual al número total de las celdas de referencia (128), $N_{gc}$ es igual al número total de las celdas de guarda (10) y $N_t$ es igual a el número total de las celdas bajo testeo (1).

\end{itemize}

\begin{equation}
T_{clk} *(N_{rc} + N_{gc} + N_t) =  0,8 us * 139 = 111,2 us
\label{Eq:delay_cfar}
\end{equation}



Se introdujo dos señales de reloj de diferentes periodos, de valores similares a los periodos reales. El primero para comandar la transferencia de datos entre las celdas CFAR y el segundo para realizar la comparación entre el umbral generado por el CFAR y el valor de video de la celda bajo testeo.

Se adoptó como valor inicial de multiplicador el valor $15784$ debido a que en ensayos preliminares con señal de radar real se observó que el mismo era un valor apropiado para determinar un rango útil de multiplicador.

En resumen, se introdujo los siguientes estímulos:
\begin{itemize}
\item Señal de reloj 1 de 800 ns de periodo.
\item Señal de reloj 2 de 20 ns de periodo.
\item Señal de algoritmo en "00". Correspondiente al algoritmo CA-CFAR.
\item Señal de video de entrada \texttt{video\_cf\_tb} de \texttt{14\-bit} tipo unsigned, periódica de periodo igual a 61 us, con valor \texttt{10000000000000} (8192 en sistema decimal) durante 60 us y \texttt{10111011100000} (12000 en sistema decimal) durante 1 us.
\item Señal de habilitación con valor permante en '1' lógico (siempre habilitado).
\item Señal de reset con valor permanente en '0' lógico (nunca resetea).
\item Señal de multiplicador inicializado en 15784 con incremento de una unidad en cada flanco ascendente del reloj 1.
\end{itemize}

El resultado de este ensayo se ilustra en la figura \ref{fig:cfar_ensayo_1}. Un acercamiento de la señal de multiplicador se ilustra en la figura \ref{fig:cfar_ensayo_1_zoom_multiplicador}.

\begin{figure}
\centering
\includegraphics[scale=0.52, angle=270]{./Figures/cfar_ensayo_1.png}
\caption{CFAR. Ensayo 1 con \textit{testbench}}
\label{fig:cfar_ensayo_1}
\end{figure}

\begin{figure}
\centering
\includegraphics[scale=1]{./Figures/cfar_ensayo_1_zoom_multiplicador.png}
\caption{Acercamiento del ensayo 1. Variación del multiplicador}
\label{fig:cfar_ensayo_1_zoom_multiplicador}
\end{figure}



Se observó lo siguiente:

\begin{itemize}
\item
Las señales de estímulos se generaron adecuadamente.

\item
% (0.8*N)-0.4 = delay. N es el número de flanco.
Desde $0$ a $56,42 us$ la salida de la celda test estuvo indefinida. Esto se debe a la cantidad de flancos de reloj necesarios del reloj 1 para que el dato llegue a la celda  test, debiendo pasar antes por 64 celdas de referencia, 5 celdas de guarda y la propia celda test. Una vez que se actualizó la salida del la celda test se añade otro retraso interno al cfar, de un periodo del reloj 2, dado por el componente comparador, como se ilustra en la figura \ref{fig:cfar_ensayo_1_zoom_test_cell}. Este retraso debido al comparador se debe al tiempo que invierte el componente en realizar los cálculos de promediación y comparación para determinar si el video almacenado en la celda bajo testeo supera o no el umbral, la actualización de los datos de entrada y salida de este componente son dados en los flancos descendentes del reloj 2 para crear un desfase de $180º$ respecto del reloj 1 y asegurar la estabilidad del dato. Por último, se suma un retardo adicional de un periodo del reloj 1, correspondiente al registro de salida del cfar.

\begin{figure}
\centering
\includegraphics[scale=1]{./Figures/cfar_ensayo_1_zoom_test_cell.png}
\caption{CFAR.Ensayo 1. Acercamiento al tiempo $t=56.4 us$}
\label{fig:cfar_ensayo_1_zoom_test_cell}
\end{figure}


\item
Las señales de salida \textit{test\_mutiplicated\_cf\_test\_tb} y \textit{o\_test\_mutiplicated\_cfar}, actualizaron sus valores simultáneamente. El primero corresponde a la salida de la celda test; se actualizó al valor \texttt{134217728}, correspondiente a una corrección dada por la ecuación \ref{Eq:corrección_test_cell}, donde: $V_{tc corregido}$ es la corrección sobre el valor de video almacenado en la celda test, $V_{tc}$ es el valor de video almacenado en la celda test y $2^{14}$ es la resolución del ADC de alta velocidad. La segunda señal corresponde al puerto de salida del cfar de la celda test; se actualizó al valor \texttt{1000} (en sistema binario) debido a que representa en el sistema binario a una porción del valor de la celda test. Ocurre algo similar luego de $60 us$ cuando llega el pulso de video a la celda test.

\begin{equation}
V_{tc   corregido} = V_{tc} * 2^{14} = 8192 * 16384 = 134217728
\label{Eq:corrección_test_cell}
\end{equation}

\item
Desde 0 a 51,61 us la salida de la suma de todas las celdas de referencias ubicadas a la izquierda de la celda test, \texttt{sum\_rc\_l\_test\_tb}, se encuentra indefinida. Este retraso se ilustra en la figura \ref{fig:cfar_ensayo_1_zoom_sum_rc} y, como se indica en la ecuación \ref{Eq:delay_suma_rc_l}, se debe a la cantidad de flancos de reloj 1 necesarios para que el dato pase por todas las celdas y sea posible realizar la suma de sus valores almacenados, más un retraso adicional de 1 periodo del reloj 2, debido al registro de salida incorporado dentro del cfar. Luego de ese intervalo de tiempo la suma se actualizó al valor $8192 * 64 = 524288$, debido a que todas las celdas de referencia ubicadas a la izquierda de la celda test almacenaron el valor constante de $8192$. Se puede realizar un análisis similar para la celda de referencia ubicada a la derecha de la celda test, que dejó de estar indefinida a los $111,6 us$ debido al tiempo necesario para superar el retardo generado por sus celdas de referencia y las celdas anteriores, como se indica en la ecuación \ref{Eq:delay_suma_rc_r}.

\begin{figure}
\centering
\includegraphics[scale=1]{./Figures/cfar_ensayo_1_zoom_sum_rc.png}
\caption{Acercamiento del ensayo 1. Detalle de la suma de salida de la celda de referencia izquierda}
\label{fig:cfar_ensayo_1_zoom_sum_rc}
\end{figure}

\begin{equation}
T_{clk 1} * N_{rc} - \dfrac{T_clk_1}{2} + \dfrac{T_clk_2}{2} = 0,8 us * 64 - 0,4 us + 0,02 us= 56,4 us
\label{Eq:delay_suma_rc_l}
\end{equation}


\begin{equation}
0,8 us *(64 + 64 + 5 + 5 + 1) + 0,4 us = 111,6 us
\label{Eq:delay_suma_rc_r}
\end{equation}

\item
A los 111,62 us, medio periodo del reloj 2 luego de haber quedado definido la sumas de la celda de referencia izquierda y, por tanto, teniendo disponible la suma de ambas celdas de referencia, se crea el promedio normal, como se ilustra en la figura \ref{fig:cfar_ensayo_1_zoom_umbral}. El promedio normal está compuesto por la suma de ambas celdas de referencia dividido 128, valor correspondiente a la cantidad total de celdas de ambas celdas de referencia. También en la figura\ref{fig:cfar_ensayo_1_zoom_umbral} se observó que $60 ns$ después de que el promedio normal (retraso necesario para efectuar el cálculo), la señal de salida del umbral normal adquirió un valor definido. El mismo adquierió el último valor de la multiplicación del promedio por el multiplicador, como se indica en la ecuación \ref{Eq:Ecuación del umbral normal}. A partir de ese momento, su valor creció en cada flanco ascendente del reloj 1 debido a que se encuentra afectado por el multiplicador.

\begin{figure}
\centering
\includegraphics[scale=0.52, angle=270]{./Figures/cfar_ensayo_1_zoom_umbral.png}
\caption{CFAR. Ensayo 1 con \textit{testbench}, zoom alrededor de los 111,65 us}
\label{fig:cfar_ensayo_1_zoom_umbral}
\end{figure}

\begin{equation}
umbral_{normal} = promedio_{normal} * multiplicador = 8221 * 15923 = 130902983
\label{Eq:Ecuación del umbral normal}
\end{equation}

\item
Se observó a partir de los $111,68 us$ y hasta el final de la simulación ($121 us$), el nivel alto del puerto de salida del cfar correspondiente al target. Un nivel alto en un determinado instante de tiempo indica que en ese momento el valor de video almacenado en la celda test superó al umbral generado. Por tanto supondría un posible blanco.

\end{itemize}



\subsubsection{CFAR. Ensayo 2}
\label{Subsec:CFAR. Ensayo 2}

Se simuló el \textit{test bench} del CFAR para un intervalo de tiempo comprendido entre $121 us$ y $242 us$. Las señales estímulo utilizadas son las mismas que fueron descritas en las sección \ref{Subsec:CFAR. Ensayo 1}. Se ilustra el resultado de la simulación en esta porción de tiempo en la figura \ref{fig:cfar_ensayo_2}.

\begin{figure}
\centering
\includegraphics[scale=0.52, angle=270]{./Figures/cfar_ensayo_2.png}
\caption{CFAR. Ensayo 2}
\label{fig:cfar_ensayo_2}
\end{figure}


Se observó resultados similares a los descritos en la sección \ref{Subsec:CFAR. Ensayo 1}. Con diferencias en lo siguiente:

\begin{itemize}
\item
Valores definidas para todas las salidas.

\item
Valor del puerto de salida en nivel alto (1 lógico) durante todo el intervalo de tiempo de ensayo. Esto se debió a que el multiplicador se incrementa linealmente, una unidad en cada flanco ascendente del reloj 1 y hasta el final de la simulación, no adquiere el valor necesario para desplazar el umbral normal por encima del valor máximo del video de entrada.
\end{itemize}



\subsubsection{CFAR. Ensayo 3}
\label{Subsec:CFAR. Ensayo 3}

Se introdujo las señales de estímulo mencionadas en la sección \ref{Subsec:CFAR. Ensayo 1} con diferencia en la señal de video al cfar, compuesta de la siguiente manera:

\begin{itemize}
\item Señal de video de entrada \texttt{video\_cf\_tb} de \texttt{14\-bit} tipo unsigned, periódica de periodo igual a 500 us, con valor \texttt{10000000000000} (8192 en sistema decimal) durante 249.5 us, \texttt{10111011100000} (12000 en sistema decimal) durante 1 us y \texttt{10000000000000} durante 249.5 us.
\end{itemize}

Se simuló el \textit{testbench} del CFAR desde 0 us a 500 us para observar el comportamiento de las sumas de las celdas de referencia, el promedio normal y el umbral normal ante un sólo pulso en un rango de tiempo amplio. El resultado se ilustra en la figura \ref{fig:cfar_ensayo_3}.

\begin{figure}
\centering
\includegraphics[scale=0.52, angle=270]{./Figures/cfar_ensayo_3.png}
\caption{CFAR.Ensayo 3}
\label{fig:cfar_ensayo_3}
\end{figure}

Se observó una variación escalonada en las señales de salidas correspondiente a ambas sumas de las celdas de referencia y en el promedio normal. Esto fue causado por el recorrido del pulso entre las celdas del CFAR. Se observó que antes y después del paso del pulso, las sumas y el promedio normal se mantienen estables en el valor \texttt{10000000000000} (8192 en sistema decimal). La variación de las sumas se ilustra en la figura \ref{fig:sumas_celdas}. La celda de referencia ubicada a la derecha de la celda test(azul), luego de 60 us, adquiere los mismos valores que la celda de referencia ubicada a la izquierda (rojo).

\begin{figure}
\centering
\includegraphics[scale=0.5]{./Figures/cfar_ensayo_3_sumas.png}
\caption{Variación de sumas de celdas de referencia.}
\label{fig:sumas_celdas}
\end{figure}



\subsubsection{CFAR. Ensayo 4}
\label{Subsec:CFAR. Ensayo 4}

Esta simulación comprendió el intervalo de tiempo entre $0 us$ y $7000 us$. El objetivo fue observar la comportamiento de la señal de salida del CFAR correspondiente al target. El resultado se ilustra en la figura \ref{fig:cfar_ensayo_4}.

Se observó que a los $480,47 us$ el multiplicador alcanza el valor $16384$. A partir de ese momento el valor de video almacenado en la celda test superó al umbral sólo en los momentos en donde su valor correspondió al pulso de valor \texttt{10111011100000} (12000 en sistema decimal). Durante toda la simulación el valor del multiplicador se incrementó linealmente, al igual que el umbral normal.

El nivel alto del puerto de salida del CFAR, correspondiente al target, se observó con un comportamiento periódico de acuerdo a lo mencionado en el párrafo anterior. A los $6806.02 us$ la celda test presentó un valor correspondiente al pulso de video, pero no se observó nivel alto del puerto de target debido a que en ese momento el multiplicador alcanzó un valor que incrementó el umbral a un nivel superior al valor máximo del CFAR. Este valor fue $24291$. Teniendo en cuenta la expresión de la ecuación \ref{Eq:Ecuación del umbral normal} y la correción del valor almacenado en la celda bajo testeo, indicada en la ecuación \ref{Eq:corrección_test_cell}, se verificó los datos arrojados en la simulación. Estos valores se ilustran en la figura \ref{fig:cfar_ensayo_4_zoom}, la cual es una imagen ampliada de la figura \ref{fig:cfar_ensayo_4}.

\begin{figure}
\centering
\includegraphics[scale=0.52, angle=270]{./Figures/cfar_ensayo_4.png}
\caption{CFAR.Ensayo 4}
\label{fig:cfar_ensayo_4}
\end{figure}
	
\begin{figure}
\centering
\includegraphics[scale=0.7, angle=0]{./Figures/cfar_ensayo_4_zoom_multiplicador.png}
\caption{CFAR.Ensayo 4. Acercamiento al tiempo $t=6806.02 us$}
\label{fig:cfar_ensayo_4_zoom}
\end{figure}

\subsection{Resultados del Configurador a ensayos con marcos de prueba}


\subsubsection{Configurador. Ensayo 1}
\label{Subsec:Configurador. Ensayo 1}

Se simuló el \textit{test bench} del componente \textit{sectorizador}, instancia del Configurador. El intervalo de simulación estuvo entre $0 us$ y $33 us$. Se introdujo las siguientes señales de estímulo:

\begin{itemize}
\item
\texttt{i\_clk\_tb}: Señal de reloj, de periodo igual a $1 us$.

\item 
\texttt{combinacion\_s}: Señal de $5\-bit$, que incrementó su valor en cada flanco ascendente del reloj y que produjo el efecto de realizar un barrido por todos los sectores.

\item
$32$ señales de multiplicadores, una por cada entrada del sectorizador. Se generó un estímulo que introdujo un valor diferente en cada entrada para evaluar a la salida si el pase por el sector correspondiente se correspondía con la entrada configurada.

\item
$32$ señales de algoritmo, una por cada entrada del sectorizador. Su valor fue \texttt{00} (en sistema binario) para todas las entradas.

\item
$32$ señales de coeficiente de ventana deslizante, una por cada entrada del sectorizador. Su valor fue \texttt{0110} (en sistema binario) para todas las entradas.

\item
$32$ señales de selección de video CFAR, una por cada entrada del sectorizador. Su valor fue $0$ (en sistema binario) para todas las entradas.


\end{itemize}

El tiempo de simulación estuvo comprendido en un intervalo de tiempo entre $0 us$ y $17 us$ para las señales de entrada y salida del multiplicador. Se observó la correcta asignación de los valores a los correspondientes puertos de entrada. Se observó además que salida del multiplicador presentó el valor asignado a cada puerto de entrada luego de dos flancos ascendentes desde el momento en el que el sector actual coincidió numéricamente con el número del puerto de entrada. Por ejemplo, como se ilustra en la figura \ref{fig:cfar_ensayo_1_zoom_multiplicador}, la salida del multiplicador en el tiempo $t = 9 us$, correspondiente al sector actual con valor \texttt{01001} (Sector 10), adquirió el valor \texttt{0000000000001001} de la entrada correspondiente al sector 10, 2 flancos ascendentes después ($10,5 us$) debido al retraso del circuito combinacional. Un análisis similar se puede realizar para las señales restantes debido a que tanto el circuito que proporciona la salida del multiplicador, como los circuitos respectivos de selección de video, algoritmo y coeficiente de ventana deslizante, comparten una arquitectura similar.

\begin{figure}
\centering
\includegraphics[scale=0.5, angle=270]{./Figures/sectorizador_ensayo_1_zoom_mult.png}
\caption{Sectorizador. Ensayo 1. Acercamiento alrededor de $t = 9us$.}
\label{fig:sectorizador_ensayo_1_zoom_mult}
\end{figure}


\begin{itemize}
\item
Valor bajo equivalente al piso de ruido durante todo el periodo, excepto durante un pequeño intervalo de tiempo en la que adquiere un valor mayor. Se eligió como valor bajo \texttt{10000000000000} debido a que el conversor de alta velocidad es de 14-bit signado y el valor mencionado corresponde al cero. Físicamente el piso de ruido era cercano a cero. El valor de amplitud del pulso de video se eligió \texttt{10111011100000} como valor cercano al $50\%$ del rango de operación.


\item
Periódica de periodo suficientemente largo para que exista un solo pulso en el intervalo de tiempo que comprende la totalidad de celdas CFAR. La totalidad de las celdas CFAR generan un retardo indicado en la ecuación \ref{Eq:delay_cfar}, donde $N_{rc}$ es igual al número total de las celdas de referencia (128), $N_{gc}$ es igual al número total de las celdas de guarda (10) y $N_t$ es igual a el número total de las celdas bajo testeo (1).

\end{itemize}






\subsubsection{Configurador. Ensayo 2}
\label{Subsec:Configurador. Ensayo 2}
Se simuló el \textit{test bench} del componente \textit{ajuste de multiplicador}, instancia del Configurador. El intervalo de simulación estuvo entre $0 us$ y $22 us$. Se introdujo las siguientes señales de estímulo:

\begin{itemize}

\item
\texttt{i\_decision\_tb}: Señal de decisión, con valor inicial \texttt{01}, \texttt{10} luego de $3 us$, \texttt{00} luego de $13 us$ y \texttt{01} luego de $19 us$.

\item
\texttt{i\_sub\_mode\_tb}: Configuración de submodo, con valor inicial \texttt{0} y \texttt{1} luego de $2.5 us$.

\item
\texttt{i\_clk\_tb}: Señal de reloj, de periodo igual a $1 us$.

\item
\texttt{i\_rst\_tb}: Señal de reset, con valor constante $0$.

\item
\texttt{i\_multiplier\_tb}: Señal de multiplicador, con valor constante \texttt{11110000000000} ($15360$ en sistema decimal).

\end{itemize}

\begin{figure}
\centering
\includegraphics[scale=0.52, angle=270]{./Figures/multiplier_setting_ensayo_1.png}
\caption{Configurador. Ensayo 2}
\label{fig:multiplier_setting_ensayo_1}
\end{figure}

El resultado se ilustra en la figura \ref{fig:multiplier_setting_ensayo_1}. Un valor de decisión \texttt{01} significa "no realizar modificaciones al multiplicador", en otras palabras: dejar ese valor constante. Un valor de decisión \texttt{10} significa ''incrementar el valor del multiplicador en un delta''. Donde el \textit{delta} tuvo valor $1$ por defecto. Por último, un valor de decisión \texttt{00} significa ''decrementar el valor del multiplicador en un delta''. La señal del multiplicador de entrada es constante debido a que el mismo se almacena en un registro externo al módulo e interno a cada sector, se toma dicho valor como referencia.

Se observó que para $t = 1.5 us$, luego de $2$ flancos ascendentes de reloj, la salida queda definida en el valor configurado desde la entrada. Este valor se mantiene constante entre $t = 1.5 us$ y $t = 3 us$, intervalo de tiempo en el que el submodo estuvo con valor $0$ correspondiente a una operación fija. A partir de $t = 3 us$ el submodo adquirió valor $1$ indicando una operación automática, esto significa que se habilita con una modificación automática del último valor del multiplicador almacenado en el registro interno al módulo en función del valor de decisión de entrada. Se observó que entre $t = 3 us$ y $t = 13 us$, intervalo de tiempo en donde la decisión tomó valor $10$, el valor del multiplicador aumentó en una unidad en cada flanco ascendente de reloj. Para el intervalo de tiempo entre $t = 13 us$ y $t = 19 us$, donde la decisión tomó valor $00$, el valor del multiplicador disminuyó en una unidad en cada flanco ascendente de reloj. Finalmente, a partir de $t = 19 us$, con el valor de decisión $01$, el valor del multiplicador permaneció constante.



\subsubsection{Configurador. Ensayo 3}
\label{Subsec:Configurador. Ensayo 3}
Se simuló el \textit{test bench} del Configurador. El intervalo de simulación estuvo entre $0 us$ y $40 us$. Se introdujo las siguientes señales de estímulo:

\begin{itemize}
\item
Para el bloque fijo:
	\begin{itemize}
	\item
	\texttt{i\_multiplier\_fxs\_tb}: Señal de multiplicador, con valor constante de \texttt{1110000000000000}.
	
	\item
	\texttt{i\_algorithm\_fxs\_tb}: Señal de algoritmo, con valor constante de \texttt{00}.   
	
	\item  
	\texttt{i\_window\_sliding\_fxs\_tb}: Señal de coeficiente de ventana deslizante, con valor constante de \texttt{0000}.
	
	\item
	\texttt{i\_tolerancia\_fxs\_tb}: Señal de tolerancia, con valor constante de \texttt{00101}.    

	\end{itemize}

\item
Para un sector en particular:
	\begin{itemize}
  	\item
  	\texttt{i\_algortihm\_cfar\_tb}: Señal de algoritmo con valor \texttt{00} durante los primeros $11 us$, luego \texttt{10}.
  
  	\item
  	\texttt{i\_sub\_mode\_tb}: Señal de submodo con valor constante \texttt{0} (modo fijo).
  
  	%\item
  	%\texttt{i_presence_required_tb}. Comento porque no interesa para éste ensayo.
  
  	\item
  	\texttt{i\_multiplier\_cfar\_tb}: Señal de multiplicador, con valor \texttt{1110000000000000} durante los primeros $11 us$, luego \texttt{1111000000000001}.
  
  	\item
  	\texttt{i\_window\_sliding\_tb}: Señal de coeficiente de ventana deslizante, con valor \textit{0100} durante los primeros $11 us$, luego \texttt{1010}.
  
  	\item
  	\texttt{i\_sector\_addr\_tb}: Señal de identificación de sector, con valor \texttt{00101} durante los primeros $11 us$, luego \texttt{00011}.
	\end{itemize}

\item
Señales comunes:
	\begin{itemize}
	\item
	\texttt{i\_clk\_tb}: Señal de reloj, de periodo igual a $1 us$.
	
	\item 
	\texttt{combinacion\_s}: Señal de $5\-bit$, que incrementó su valor en cada flanco ascendente del reloj y que produjo el efecto de realizar un barrido por todos los sectores.
	
    
    \item
    \texttt{i\_mode\_tb}: Señal de modo de operación, con valor \texttt{0} (no sectorizado) durante los primeros $10 us$, luego \texttt{1} (sectorizado).
    
    \item
    \texttt{i\_load\_config\_tb}: Señal de carga de configuración a un sector determinado, consiste en un pulso de una duración de $1 us$. Se generó dos pulsos de carga de configuración, el primero a los $2 us$ y el segundo a los $13 us$.
    
    \item
    \texttt{i\_tolerancia\_tb}: Señal de modo de tolerancia, con valor constante \texttt{10} (2 en sistema decimal).
    
    \item
    \texttt{i\_passed\_through\_north\_tb}: Señal de paso por el norte, consiste en un pulso de una duración de $1 us$. Se utilizó esta señal para actualizar la salida de todos los sectores del modo sectorizado. Se generó dos pulsos de paso por el norte, el primero a los $7 us$ y el segundo a los $18 us$. Es necesario que estos pulsos se produzcan después de  el pulso de carga de configuración para que la salida presente el valor configurado con el pulso de carga mencionado.
	\end{itemize}

\end{itemize}



\begin{figure}
\centering
\includegraphics[scale=0.52, angle=270]{./Figures/configurador_ensayo_3.png}
\caption{Configurador. Ensayo 3}
\label{fig:configurador_ensayo_3}
\end{figure}


El resultado se ilustra en la figura \ref{fig:configurador_ensayo_3}. Se observó que durante los primeros $10 us$ la señal de modo permanece en $0$. Debido a esto la salida de los sectores con \textit{id} \texttt{00000} y \texttt{00001} son ignoradas mientras el sector actual adquiere estos valores, en cambio la salida del configurador adopta los valores de multiplicador, algoritmo, coeficiente de ventana y tolerancia del sector fijo. Esta salida tiene lugar desde el tiempo $t = 1 us$ en el flanco descendente del reloj, estando indefinida por este motivo en el momento inicial. Cabe mencionar que para asegurar la estabilidad de los datos al momento de la lectura por parte del CFAR, se eligió para la salida del configurador como flanco activo el flanco descendente. De esta manera se produce un desfase entre la actualización del dato a la salida de este módulo y la lectura del dato a la entrada del CFAR.

Luego de los $10 us$ se observó que la señal de modo conmutó a $1$. Esto produce un cambio en la salida del configurador. En modo sectorizado se ignora la salida del sector fijo y se considera la salida de los $32$ sectores restantes, proporcionando a la salida del configurador la configuración almacenada para el sector correspondiente al sector actual. Debido a que se configuró sólo dos sectores, se observó que la salida del configurador estuvo indefinida para aquellos sectores que no fueran los que se configuraron.

En $t = 21 us$, cuando el sector actual cambia de valor de \texttt{00010} y \texttt{00011}, las salidas del configurador adpotan los valores configurados para el sector \texttt{00011} de acuerdo a los estímulos generados. Una situación similar ocurre para $t = 21 us$ con el sector \texttt{00101}



\subsection{Resultados a ensayos con visores de netlist}
\label{Subsec: Resultados de Visores de netlist}

\begin{figure}
\centering
\includegraphics[scale=0.6, angle=270]{./Figures/RTL_cfar_1.png}
\caption{Diagrama RTL de la sección de entrada del módulo CFAR}
\label{fig:RTL_cfar_1}
\end{figure}


\begin{figure}
\centering
\includegraphics[scale=0.5]{./Figures/RTL_cfar_2.png}
\caption{Diagrama RTL de la sección de salida del módulo CFAR}
\label{fig:RTL_cfar_2}
\end{figure}



\begin{figure}
\centering
\includegraphics[scale=0.6, angle=270]{./Figures/RTL_cfar_rc_1.png}
\caption{Diagrama RTL de la sección de salida de una celda de referencia}
\label{fig:RTL_cfar_rc_1}
\end{figure}



\begin{figure}
\centering
\includegraphics[scale=0.6, angle=270]{./Figures/RTL_cfar_comparador_1.png}
\caption{Diagrama RTL de la sección de salida del comparador}
\label{fig:RTL_cfar_comparador_1}
\end{figure}






\begin{figure}
\centering
\includegraphics[scale=0.65, angle=270 ]{./Figures/RTL_cfg_sector_1.png}
\caption{Diagrama RTL de la sección de entrada de un sector}
\label{fig:RTL_cfg_sector_1}
\end{figure}



\begin{figure}
\centering
\includegraphics[scale=0.75]{./Figures/RTL_cfg_input_reg_1.png}
\caption{Diagrama RTL de una sección del registro de entrada interno a un sector}
\label{fig:RTL_cfg_input_reg_1}
\end{figure}




Se compiló el diseño del Procesador Monoradar, el cual incluyó como subsistemas a los módulos CFAR y Configurador. Superada la etapa de análisis y síntesis, fue posible utilizar la herramienta \textit{RTL Viewer} para verificar la interpretación de la herramienta de síntesis, del código RTL compilado.

Se observó el esquema RTL de las celdas CFAR, el cual se ilustra en la figura \ref{fig:RTL_cfar_1}. Los registros de desplazamientos se conectaron adecuadamente. Se verificó que todos fueron conectados a la misma fuente de reloj y compartieron las señales de habilitación y reset. Estas características permiten, según el fabricante, una adecuada inferencia de los registros de desplazamiento como tales y por tanto permiten que se efectúen procesos de optimización sobre el diseño. Se verificó además que exista una conexión serie entre estos registros. En cuanto a la sección de salida del CFAR, se verificó las conexiones del comparador y la interpretación, por parte de la herramienta de síntesis, de la lógica combinacional y secuencial de salida. Se ilustra en la figura \ref{fig:RTL_cfar_2}, los registros y \textit{flip flop} inferidos por la herramienta de síntesis a la salida del módulo CFAR. La herramienta de síntesis interpretó una lógica intermedia entre las señales provenientes de las celdas CFAR o el comparador y los registros de salida, para las señales de targets y de suma. Esta lógica consistió en una compuerta \texttt{AND} esquematizada con un multiplexor.

Desplegando uno de los módulos correspondientes a las celdas de referencia, se observó la estructura interna de esos módulos. La sección de salida se ilustra en la figura \ref{fig:RTL_cfar_rc_1}. Se observó una conexión serie de los registros, una conexión de sumadores en cascada y dos registros de salida para almacenar la suma y el video. En particular la conexión de sumadores en cascada permitió la escalabilidad del módulo; fue posible durante el diseño generar celdas de referencia con diferente cantidad de celdas sin mayores complicaciones.

Para el componente \textit{comparador} se observó la interpretación por parte de la herramienta de síntesis de un circuito combinacional y otro secuencial, acorde a la descripción realizada en la sección \ref{metodologia_estructurada} del capítulo \ref{Chapter2} sobre la \textit{metodología estructurada}. En la figura \ref{fig:RTL_cfar_comparador_1} se ilustra la sección de salida del comparador compuesta por un circuito secuencial.

En cuanto al configurador, se verificó la interpretación de la descripción RTL del módulo. Se ilustran ejemplos de los esquemas RTL del sector y del registro de entrada interno al sector, en las figuras \ref{fig:RTL_cfg_sector_1} y \ref{fig:RTL_cfg_input_reg_1}. En la figura \ref{fig:RTL_cfg_sector_1} se ilustra la condición de igualdad que debe cumplirse entre la señal de combinación de sector y la señal de identificación de ese sector, luego con una compuerta \textit{AND} se impuso otra condición para que se habilite la carga de la configuración con una señal adicional (proveniente del puente HPS). En la figura \ref{fig:RTL_cfg_input_reg_1} se ilustra parte de los diferentes componentes que componen el módulo: por un lado \textit{flip flops} tipo D para la selección del video CFAR y el submodo, y por el otro registros para la presencia requerida, algoritmo y multiplicador.




\subsection{Resultados a ensayos con simulador y decodificador}
\label{Subsec: Resultados de simulador y decodificador}

Debido a que el módulo configurador no necesitó de señal de radar real para operar, se utilizó un decodificador y un simulador para verificar su funcionamiento luego del proceso de implementación en la FPGA, como se mencionó en la sección \ref{Ensayos_simulador_decodificador} del capítulo \ref{Chapter3}. El decodificador recibió las señales de salida del configurador correspondientes a:

\begin{itemize}
\item Señal proveniente del puerto de salida del algoritmo
\item Señal proveniente del puerto de salida del coeficiente de ventana deslizante
\item Señal proveniente del puerto de salida del multiplicador
\end{itemize}

A su vez, en el interior del decodificador se almacenó una configuración para estos parámetros más una configuración adicional para otro valor de multiplicador. La configuración almacenada fue:

\begin{itemize}
\item Algoritmo: \texttt{01}
\item Multiplicador 1: \texttt{1111111111111110} (65534 en sistema decimal)
\item Multiplicador 2: \texttt{1000000000000000} (32768 en sistema decimal)
\item Coeficiente de ventana deslizante: \texttt{0011} (3 en sistema decimal)
\end{itemize}


Luego, en caso de que alguna señal de entrada del decodificador coincidiera con el valor almacenado de ese parámetro, se debía emitir un nivel lógico alto de salida. Este \textit{bit} se utilizaba para comandar cuatro \textit{leds} indicadores incorporados a la placa ADC-SoC. De esta, manera por ejemplo, cuando la señal de salida de coeficiente de ventana sea \texttt{0011} se debía encender el led correspondiente a ese parámetro, de lo contrario debía estar apagado. Además se conectó la señal de paso por el norte a otro led adicional, para visualizar el momento en el que se completaba una vuelta de radar.

Con las configuración almacenada se ejecutó el simulador de blancos implementado en la placa ADC-SoC. La parte del simulador de interés para este ensayo fue el barrido en rango y azimut, que permitió para que el configurador realice un barrido por todos los sectores implementados. Se configuró la cobertura con los dos valores almacenados en el decodificador. Se dividió la cobertura en dos, a la mitad del intervalo total de azimut. Una mitad de la cobertura (16 sectores) se configuró con con un valor de multiplicador \texttt{1111111111111110} y la otra con valor \texttt{1000000000000000}. Luego se configuró toda la cobertura con los valores de algoritmo y coeficiente de ventana deslizante almacenados en el decodificador.

Con el simulador y el procesador monoradar en ejecución, y después de realizada la configuración, se observó el encendido y apagado de los leds. Cuando el simulador completó una vuelta de radar el led de paso por el norte produjo un centelleo (breve encedido); a partir de ese momento y durante la mitad del tiempo de cobertura se encendió el led de multiplicador 1 mientras el led de multiplicador 2 estuvo apagado. Luego desde la mitad del tiempo de cobertura hasta el final de la misma se encendió el led de multiplicador 2 mientras el led de multiplicador 1 estuvo apagado. Los leds correspondientes al algoritmo y al coeficiente de ventana deslizante estuvieron encendidos sólo durante la mitad del tiempo de cobertura debido a que sólo una mitad de la cobertura se configuró para cumplir con la condición del decodificador.

\subsection{Resultados a ensayos con procesador monoradar preexistente}
\label{Subsec: Resultados a procesador monoradar preexistente}

Se realizaron pruebas con un radar FPS113 de la Fuerza Aérea Argentina. Se utilizó un procesador monoradar preexistente para verificar el funcionamiento del CFAR luego del proceso de implementación en la FPGA, como se mencionó en la sección \ref{Ensayos_Procesador_Monoradar_preexistente} del capítulo \ref{Chapter3}. No se empleó el simulador debido a que para verificar el funcionamiento del CFAR fue necesario contar con señal de radar real.

Se utilizó un decodificador para visualizar con unos leds incorporados en la placa ADC-SoC la cuenta de targets del CFAR en una vuelta de antena. Esto se llevó a cabo con el módulo \textit{contador CFAR} el cual consistió en un acumulador que proporcionaba 4 salidas de 1-bit correspondiente a los 4 bits más significativos de la cuenta. Se verificó que para en cada vuelta de antena se obtiene una cuenta ascendente de targets que se reseteaba en la vuelta de antena siguiente.

Se empleó el procesador monoradar preexistente para realizar una comparación con el nuevo procesador monoradar del cual los subsistemas diseñados en este trabajo fueron parte. Se verificó que los blancos detectados por ambos sistemas coincidían en valores de rango y azimut, lo que significó que para la misma señal de entrada se produjo resultados similares. 
% Chapter Template

\chapter{Conclusiones} % Main chapter title

\label{Chapter5} % Change X to a consecutive number; for referencing this chapter elsewhere, use \ref{ChapterX}


%----------------------------------------------------------------------------------------

%----------------------------------------------------------------------------------------
%	SECTION 1
%----------------------------------------------------------------------------------------

Se diseñó e implementó un sistema CFAR para un Procesador Monoradar capaz de elaborar umbrales adaptativos de acuerdo los algoritmos CA-CFAR (\textit{Cell Averaging CFAR}), SOCA-CFAR \textit{Smallest Of Cell Average CFAR} y GOCA-CFAR \textit{Greatest Of Cell Average CFAR}.

Se diseñó e implementó un sistema configurador capaz de sectorizar la cobertura en 32 sectores. De esta manera se logró mediante calibración poder optimizar, para cada sector, los valores de configuración utilizados por el filtro CFAR, así como otros valores usados en etapas posteriores de procesamiento.

El diseño de ambos sistemas fue realizado utilizando el lenguaje de descripción de hardware VHDL. La implementación de estos diseños se llevó a cabo en una placa ADC-SoC de la firma \textit{TerasIC}.

Se realizaron pruebas con un radar FPS113 de la Fuerza Aérea Argentina y el sistema cumple exitosamente con los requerimientos. Esta implementación introdujo mejoras al procesamiento de señales. El trabajo en un futuro podría continuar con otras funciones automáticas que calibren el sistema sin depender de un operador.

 
Para el presente trabajo, la extensión final de la descripción de los módulos (sin contar diseños previos no utilizados ni \textit{testbenchs}) fue de 2200 líneas. Con las descripciones en VHDL se se realizaron alrededor de 280 contribuciones (\textit{commits}) al repositorio utilizado por el equipo formado para el diseño e implementación del Procesador Monoradar. En la implementación final del Procesador Monoradar se utilizaron alrededor de 7400 módulos lógicos adaptativos (ALM) de la FPGA. 

%----------------------------------------------------------------------------------------
%	CONTENIDO DE LA MEMORIA  - APÉNDICES
%----------------------------------------------------------------------------------------

\appendix % indicativo para indicarle a LaTeX los siguientes "capítulos" son apéndices

% Incluir los apéndices de la memoria como archivos separadas desde la carpeta Appendices
% Descomentar las líneas a medida que se escriben los apéndices

%\include{Appendices/AppendixA}
%\include{Appendices/AppendixB}
%\include{Appendices/AppendixC}

%----------------------------------------------------------------------------------------
%	BIBLIOGRAPHY
%----------------------------------------------------------------------------------------

\Urlmuskip=0mu plus 1mu\relax
\raggedright
%\printbibliography[heading=bibintoc]

%----------------------------------------------------------------------------------------

\end{document}  
